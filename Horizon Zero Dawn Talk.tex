
\documentclass[a4paper]{article}

%% Language and font encodings
\usepackage[english]{babel}
\usepackage[utf8x]{inputenc}
\usepackage[T1]{fontenc}

%% Sets page size and margins
\usepackage[a4paper,top=2cm,bottom=2cm,left=3cm,right=3cm,marginparwidth=1.75cm]{geometry}

%% Useful packages
\usepackage{amsmath}
\usepackage{graphicx}
\usepackage[colorinlistoftodos]{todonotes}
\usepackage[colorlinks=true, allcolors=blue]{hyperref}

\providecommand{\keywords}[1]{\textbf{\textit{Tags:}} #1}
\providecommand{\talkurl}[1]{\textbf{\textit{Url:}} #1}
\providecommand{\track}[1]{\textbf{\textit{Track:}} #1}
\providecommand{\speaker}[1]{\textbf{\textit{Speaker:}} #1}



\title{'Horizon Zero Dawn' : A Game Design Postmortem \\by Eric Boltjes}
\author{Author: Bastien Pery}

\begin{document}
\maketitle

\begin{keywords} Horizon Zero Dawn, Game Design, Design decisions, Ambitious project \end{keywords}

\begin{track} GDC EXPO San Francisco March 2018 - Game Design \end{track}

\begin{talkurl}  \url{https://www.gdcvault.com/play/1024963/-Horizon-Zero-Dawn-A} \end{talkurl}

\begin{speaker}Eric Boltjes, Guerrilla Games \end{speaker}


\begin{abstract}


\end{abstract}

\section{Summary of Talk}

%%Your summary of the talk goes here! (in your own words!) 
%%Describe the main points / lessons learned of the talk, the relevance for game development. 

Through the creation of early prototypes and design decisions or processes that shaped
development of Horizon Zero Dawn, this talk takes us into the journey game design went through while 
moving from an ambitious paper concept to a finished open world action RPG. Thanks to Eric Boltjes'
details, this talk helps us to understand how small and large design decisions and choices can 
improve a game and players' experience.

\subsection{Early Concept}

Guerrilla Games studio never did an open world video game before. Here was for them the challenge
represented by Horizon. Indeed, the idea of the game came in 2001 with some important guidelines
such as :

\begin{itemize}
  \item Majestic post-apocalyptic wilderness
  \item Awe-inspiring machines (as an innovant concept)
  \item Exotic tribes
  \item And the open world idea 
\end{itemize}

\noindent But like all beginners, they didn't know how to structure that kind of game development. So,
they started with a really small team in 2011 composed by designers, artists, coders and 
animators. Each of them without a specific role. And, of course, with a lot of questions
(how the open world is going to look like, what kind combat against machines and so on).\\

\noindent As they didn't have a global idea of how everything will work together, they decided to
do different prototypes to test them and judge their reliability. The feedbacks were really
good in the sense that these prototypes were really effective and helped them to really know
what they wish to create for Horizon Zero Dawn. But it was costly in money and time. And also, 
prototype after prototype, the team started to think "Where is the game so?".\\

\noindent Thus, they answered two different fundamental questions before starting the pre-production:
\begin{itemize}
  \item What Horizon Zero Dawn is
  \item What it is NOT
\end{itemize}

\subsection{Pre-Production}

As they started to know where they were going, they focused on three major parts of the game for
the pre-production step:
\begin{itemize}
  \item World systems and mechanics
  \item World building
  \item Horizon Zero Dawn story
\end{itemize}

\noindent They so improved the size of the team by adding new responsabilities in it such as Core Design,
World Design, Story Design etc. Especially for the story. They didn't really thought about it
during the concept step. Everything still needed to be done.\\

\noindent They also worked to answer a lot of different questions they had. At this point, the team
still needed a context to be able to have the full game idea in their mind. So, they focused on
different parts of the game which still needed to be improved.
\begin{itemize}
  \item Riding (they wanted a special character so the normal horse has been replaced by a machine)
  \item The main character, Alloy (personality, style, gameplay...)
  \item Machines (behaviors, look and also "feelings")
\end{itemize}

\noindent Fixed contexts helped the team a lot. But in game development think about don't do it too 
soon or it will prevent cool ideas to come. Everything went well so they thought that they had enough
answered to start the production in a good work environment. Unfortunately, problems started to rise.






\subsection{Production}
\subsection{Polishment}

\section{Overview and Relevance}
Research on the topic of the talk; overall overview and the relevance of the technologies/techniques; give a short overview on the state of the art of the topic, reference further readings and current developments. 

Provide a list of further readings, links (websites, papers, talks, articles,...) in the bibliography  

\subsection{Subsection..}
\subsection{Subsection..}

\renewcommand{\refname}{\section{References and Further Sources}}
\begin{thebibliography}{1}

\bibitem{lamport94}
  Leslie Lamport,
  \emph{\LaTeX: a document preparation system},
  Addison Wesley, Massachusetts,
  2nd edition,
  1994.

\end{thebibliography}

\end{document}